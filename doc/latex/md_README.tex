{\ttfamily Kolmogorov41} is an open source hybrid parallel C++ code to compute structure functions for a given velocity or scalar field.

\subsection*{Getting the Source Code}

{\ttfamily Kolmogorov41} is hosted on Git\-Hub. You can download the source code from the following link\-:

\href{https://github.com/ShubhadeepSadhukhan1993/Kolmogorov41}{\tt https\-://github.\-com/\-Shubhadeep\-Sadhukhan1993/\-Kolmogorov41}

\subsection*{Installing {\ttfamily Kolmogorov41}}

\subsubsection*{Required Libraries}

The following libraries are required for installing and running Kolmogorov41\-:


\begin{DoxyEnumerate}
\item {\ttfamily C\-Make}
\item {\ttfamily Blitz++}
\item {\ttfamily Y\-A\-M\-L-\/cpp}
\item {\ttfamily M\-P\-I\-C\-H}
\item {\ttfamily H\-D\-F5}
\item {\ttfamily H5\-S\-I}
\end{DoxyEnumerate}

The instructions to download and install these libraries are provided in the following website\-:(\href{http://turbulencehub.org/index.php/codes/tarang/installing-tarang/}{\tt http\-://turbulencehub.\-org/index.\-php/codes/tarang/installing-\/tarang/}).

\subsubsection*{Compiling instruction}

After downloading {\ttfamily Kolmogorov41}, change into {\ttfamily Kolmogorov41-\/master/src} directory and run the command {\ttfamily make} in the terminal. An executable named {\ttfamily Kolmogorov41.\-out} will be created inside the {\ttfamily Kolmogorov41-\/master/src} folder.

\subsection*{Testing {\ttfamily Kolmogorov41}}

{\ttfamily Kolmogorov41} offers an automated testing process to validate the code after installation. The relevant test scripts can be found in the {\ttfamily tests/} folder of the code. To execute the tesing process, change into {\ttfamily \textbackslash{}Kolmogorov41-\/master} and run the command {\ttfamily bash run\-Test.\-sh}. The code then runs two test cases; these are as follows.


\begin{DoxyItemize}
\item In one case, the code will generate a 2\-D velocity field given by {\bfseries u} = \mbox{[}{\itshape x, y, z}\mbox{]}, and compute the structure functions for the given field. For this case, the longitudinal structure functions should equal {\itshape l$^{\mbox{q}}$ }.
\item In the second case, the code will generate a 2\-D scalar field given by {\itshape T = x + y + z}, and compute the structure functions for the given field. Fir this case, the structure functions should equal $\ast$(l$_{\mbox{x}}$  + l$_{\mbox{z}}$ )$^{\mbox{q}}$ $\ast$.
\end{DoxyItemize}

For both the cases, {\ttfamily Kolmogorov41} will compare the computed structure functions with the analytical results. If the percentage difference between the two values is less than 10$^{\mbox{-\/10}}$ , the code is deemed to have passed.

Finally, for visualization purpose, the python script {\ttfamily test/test.\-py} is invoked. This script generates the plots of the second and third-\/order longitudinal structure functions versus {\itshape l}, and the density plots of the computed second-\/order scalar structure functions and $\ast$(l$_{\mbox{x}}$  + l$_{\mbox{z}}$ )$^{\mbox{2}}$ $\ast$. These plots demonstrate that the structure functions are computed accurately. Note that the following python modules are needed to run the test script successfully\-:


\begin{DoxyEnumerate}
\item {\ttfamily h5py}
\item {\ttfamily numpy}
\item {\ttfamily matplotlib}
\end{DoxyEnumerate}

\subsection*{Running {\ttfamily Kolmogorov41}}

{\ttfamily Kolmogorov41-\/master} has a folder named {\ttfamily in}. This folder contains the input field files in {\ttfamily hdf5} format, and a parameters file named {\ttfamily para.\-yaml}. You need to provide the required input parameters in this file. The details of the entries are as follows\-:

\subsubsection*{i) {\ttfamily para.\-yaml} details}

\paragraph*{{\ttfamily program\-: grid\-\_\-switch}}

You can enter {\ttfamily true} or {\ttfamily false}

{\ttfamily true}\-: Save the structure function output as a function of the difference vector ({\bfseries l}) in addition to the magnitude of the difference vector ({\itshape l}).

{\ttfamily false}\-: Save structure functions as a function of the magnitude of the difference vector ({\itshape l}) only.

\paragraph*{{\ttfamily program\-: scalar\-\_\-switch}}

You can enter {\ttfamily true} or {\ttfamily false}

{\ttfamily true}\-: Calculate the structure function of a scalar field.

{\ttfamily false}\-: Calculate the structure function of a vector field.

\paragraph*{{\ttfamily program\-: 2\-D\-\_\-switch}}

You can enter {\ttfamily true} or {\ttfamily false}.

{\ttfamily true}\-: Calculate the structure function for two dimensional field.

{\ttfamily false}\-: Calculate the structure function for three dimensional field.

\paragraph*{{\ttfamily program\-: Only\-\_\-logitudinal}}

This entry is for structure function of velocity fields only. You can enter {\ttfamily true} or {\ttfamily false}.

{\ttfamily true}\-: Compute only the longitudinal structure function.

{\ttfamily false}\-: Compute both longitudinal and transverse structure functions.

\paragraph*{{\ttfamily program\-: Number\-\_\-of\-\_\-\-Open\-M\-P\-\_\-processors}}

Enter the number of Open\-M\-P processors.

\paragraph*{{\ttfamily grid\-: Nx, Ny, Nz}}

The number of points along {\itshape x}, {\itshape y}, and {\itshape z} direction respectively of the grid. Valid for both the vector and scalar fields. For two dimensional fields you need to provide {\ttfamily Nx} and {\ttfamily Nz}.

\paragraph*{{\ttfamily domain\-\_\-dimension\-: Lx, Ly, Lz}}

Length of the cubical box along {\itshape x}, {\itshape y}, and {\itshape z} direction respectively. For two dimensional fields, you need to provide {\ttfamily Lx} and {\ttfamily Lz}.

\paragraph*{{\ttfamily structure\-\_\-function\-: q1, q2}}

The lower and the upper limit of the order of the structure functions to be computed.

\paragraph*{{\ttfamily test\-: test\-\_\-switch}}

{\ttfamily true}\-: For running in test mode. Idealized velocity and scalar fields are generated internally by the code. Computed structure functions are compared with analytical results. The code is P\-A\-S\-S\-E\-D if the percentage difference between the two results is less than {\ttfamily 1e-\/10}.

{\ttfamily false}\-: The \char`\"{}regular\char`\"{} mode, in which the code reads the fields from the hdf5 files in the {\ttfamily in} folder.

\subsubsection*{ii) Files Required\-:}

All the files storing the input fields should be inside the {\ttfamily in} folder.

\paragraph*{For two dimensional fields}

For vector field, two files named as {\ttfamily U.\-V1r.\-h5} and {\ttfamily U.\-V3r.\-h5} are required. Each file has one dataset.

For scalar field, one file named as {\ttfamily T.\-Fr.\-h5} is required. Each file has one dataset.

Size of the array stored in these files should be ({\ttfamily Nx,Nz}).

{\itshape Important\-:} Dataset name should be the same as the file name. For example, the dataset inside the file {\ttfamily U.\-V1r.\-h5} should be named {\ttfamily U.\-V1r}.

\paragraph*{For three dimensional fields}

For vector field, three files named as {\ttfamily U.\-V1r.\-h5}, {\ttfamily U.\-V2r.\-h5}, and {\ttfamily U.\-V3r.\-h5} are required. Each file has one dataset.

For scalar field, one file named as {\ttfamily T.\-Fr.\-h5} is required. Each file has one dataset.

Size of the array stored in these files should be ({\ttfamily Nx, Ny, Nz}).

{\itshape Important\-:} Dataset name should be the same as the file name. For example, the dataset inside the file {\ttfamily U.\-V1r.\-h5} should be named {\ttfamily U.\-V1r}.

\subsubsection*{iii) Running Instructions}

Open the terminal change into {\ttfamily Kolmogorov41-\/master/in} folder. Open {\ttfamily para.\-yaml} to set all the parameters. Keep all the required files compatible with the parameter file. Now, move out of the {\ttfamily in} folder run the command

{\ttfamily mpirun -\/np \mbox{[}number of M\-P\-I processors\mbox{]} src/\-Kolmogorov41.\-out}

\subsubsection*{iii) Output Information}

\paragraph*{a) If {\ttfamily grid\-\_\-switch} is set to {\ttfamily false}\-:}

{\bfseries Velocity structure functions}\-:

The logitudinal structure functions of order {\ttfamily q1} to {\ttfamily q2} are stored in the files {\ttfamily S\-F.\-h5} and {\ttfamily S\-F\-\_\-perp.\-h5} respectively as two dimensional arrays. Here, the first index is for different {\itshape n}, which ranges from 0 to {\itshape Nl}, where {\itshape Nl} is the number of gridpoints along the diagonal of the domain. The second index is for the order.

{\bfseries Scalar structure functions}\-:

The structure functions of order {\ttfamily q1} to {\ttfamily q2} are stored in the files {\ttfamily S\-F.\-h5} as two dimensional array. Here, the first index is for different {\itshape n}, which ranges from 0 to {\itshape Nl}, where {\itshape Nl} is the number of gridpoints along the diagonal of the domain. The second index is for the order.

\paragraph*{b) If {\itshape grid\-\_\-switch} is set to {\ttfamily true}}

{\bfseries Velocity structure functions}\-:

The logitudinal and transverse structure functions of order {\ttfamily q} are stored in the files {\ttfamily S\-F\-\_\-\-Grid\-\_\-pll}+{\ttfamily q}+{\ttfamily .h5} and {\ttfamily S\-F\-\_\-\-Grid\-\_\-perp}+{\ttfamily q}+{\ttfamily .h5} respectively as two/three dimensional arrays for two/three dimensional input fields.

Note\-: If you only want the logitudinal structure function then it will store the data for positive {\ttfamily lz} only as it saves computation time and computer memory

{\bfseries Scalar structure functions}\-:

The structure functions of order {\ttfamily q} are stored in the files {\ttfamily S\-F\-\_\-\-Grid\-\_\-pll}+{\ttfamily q}+{\ttfamily .h5} as two/three dimensional arrays for two/three dimensional input fields.

\subsection*{License}

Kolmogorov41 is released under the terms of B\-S\-D New License. 